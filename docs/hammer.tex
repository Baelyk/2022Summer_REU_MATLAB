\documentclass{article}

\usepackage[utf8]{inputenc}
\usepackage{amsmath,amsfonts,fullpage,amsthm,amssymb}
\usepackage{thmtools}
\usepackage{enumitem}

% Better matrices, namely augmented
% https://tex.stackexchange.com/a/2244
\makeatletter
\renewcommand*\env@matrix[1][*\c@MaxMatrixCols{c}]{%
	\hskip -\arraycolsep
	\let\@ifnextchar\new@ifnextchar
	\array{#1}}
\makeatother

\usepackage{natbib}
\usepackage{graphicx}
\usepackage{ifthen}
% Code blocks
\usepackage[outputdir=latex-build]{minted}

\usepackage[table]{xcolor}

% TikZ for Graphs
\usepackage{tikz}
\usepackage{tikz-3dplot}
\usetikzlibrary{positioning, quotes}

% Defining shortcut macros
\newcommand\Z{\mathbb{Z}}
\newcommand\R{{\mathbb{R}}}
\newcommand\Q{{\mathbb{Q}}}
\newcommand\C{{\mathbb{C}}}
\newcommand\N{{\mathbb{N}}}
\newcommand\F{{\mathcal{F}}}
\newcommand\TR{{\operatorname{tr}}}
% Replace \P make paragraph symbol with \P makes \mathcal{P} for power sets
\renewcommand\P{{\mathcal{P}}}
\newcommand{\then}{\Rightarrow}

\DeclareMathOperator{\im}{Im}
\DeclareMathOperator{\arccot}{arccot}

\declaretheoremstyle[
	spaceabove={1em plus 0.1em minus 0.2em},
	notefont=\bfseries,
	notebraces={}{},
	headformat=\ifthenelse{\equal{\NOTE}{}}{\NAME{} \NUMBER}{\!\!\NOTE}
]
{thm}

\declaretheorem{theorem}
\declaretheorem[name=Farka's Lemma, numbered=unless unique]{farkas}
\declaretheorem[style=remark]{example}
\declaretheorem{lemma}

\declaretheoremstyle[
	spaceabove={1em plus 0.1em minus 0.2em},
	notefont=\bfseries,
	notebraces={}{},
	headformat=\ifthenelse{\equal{\NOTE}{}}{\NAME{} \NUMBER}{\!\!\NOTE}
]
{problem}

\declaretheorem{problem}[style=problem]

\declaretheoremstyle[
	headfont=\itshape,
	bodyfont={
			% spacing inbetween paragraphs (with rubber padding)
			\parskip 0.3em plus 0.1em minus 0.2em
			% disables paragraph indenting
			\parindent 0pt
			% no obnoxious space before lists
			\setlist{nosep}
			% indent description lists
			\setlist[description]{labelindent=1.4em,labelwidth=!,format=\normalfont\itshape}
		}
]
{solution}
\declaretheorem{solution}[numbered=no, style=solution]

\newcommand{\solpart}[1]{\par\textbf{#1}  }

\title{The chordarc constant need not involve a vertex}
\author{George Ekman}

\begin{document}
%\maketitle

\begin{lemma}
	The chordarc constant does not to occur with a pair of points where one of the points is a vertex,
	even when the pair are on nonadjacent edges.
\end{lemma}
\begin{figure}[h]
	\centering
	\begin{tikzpicture}
		\draw (0, 0)
		-- (2, 0) node[midway, below = 1pt] {$2$}
		-- (2, 0.5)
		-- (2 + 0.35 * 10, 0.5) node[circle, fill, inner sep = 1pt, label=below:$a$] {}
		-- (12, 0.5)
		-- (12, 1.5) node[midway, right = 1pt] {$W$}
		-- (2 + 0.35 * 10, 1.5) node[circle, fill, inner sep = 1pt, label=above:$b$] {}
		-- (2, 1.5)
		-- (2, 2)
		-- (0, 2)
		-- (0, 0) node[midway, left = 1pt] {$2$};

		\draw (2, 0.5) -- (12, 0.5) node[midway, below = 1pt] {$L$};
	\end{tikzpicture}
	\caption{A hammer with $L = 10$, $W = 1$}
	\label{fig:hammer}
\end{figure}

\begin{proof}
	The hammer in Figure~\ref{fig:hammer} is a polygon, so the largest possible arc distance is half
	the perimeter of the polygon, and occurs when the distance is the same measuring in either
	direction, either through the head or handle of the hammer. The smallest euclidean distance between
	points $a,b$ on nonadjacent edges is $W$, when $a,b$ are on the top and bottom of the handle with
	the same $x$ coordinate.

	As long as the length of the handle is strictly greater than the perimeter of the head, this
	distance will occur along the handle, so we can parameterize the $x$ coordinate of $a,b$ in terms
	of $\lambda \in (0, 1)$: $2 + \lambda L$. Since $\lambda \in (0, 1)$, neither of $a,b$ is a vertex.
	Then, the maximum arc distance between $a,b$ of this form will occur when the distance going
	through the head and going through the end of the handle is the same:
	\begin{align*}
		l_\text{head}(a,b) &= l_\text{handle}(a,b) \\
		2 \lambda L + 8 - W &= 2 (1 - \lambda) L + W \\
		4 \lambda L &= 2L + 2W - 8 \\
		\lambda &= \frac{L + W - 4}{2L}
	\end{align*}
	This formula for $\lambda$ shows that there is only one pair of points on nonadjacent edges that
	have both minimum euclidean distance and maximum arc distance.

	Now, it remains to show that an adjacent pair of edges do not have a greater chordarc. Since all
	adjacent edges meet at right angles, the Adjacent Chord Arc gives
	\[ h(\theta) = \frac{\sqrt{2}}{\sqrt{1 - \cos\theta}} = \sqrt{2} \text{.} \]
	Let $L = 10$ and $W = 1$. Then, $\lambda = 0.35$, and the chordarc between $a,b$
	\[ \frac{l(a,b)}{\Vert a - b \Vert} = \frac{2 \lambda L + 8 - W}{W} = 14 \text{.}\]
	Therefore, the chordarc constant of the hammer is $14$, and is not reached at a pair including a
	vertex.
\end{proof}

\end{document}
