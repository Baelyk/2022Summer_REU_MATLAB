\documentclass{article}

\usepackage[utf8]{inputenc}
\usepackage{amsmath,amsfonts,fullpage,amsthm,amssymb}
\usepackage{thmtools}
\usepackage{enumitem}

% Better matrices, namely augmented
% https://tex.stackexchange.com/a/2244
\makeatletter
\renewcommand*\env@matrix[1][*\c@MaxMatrixCols{c}]{%
	\hskip -\arraycolsep
	\let\@ifnextchar\new@ifnextchar
	\array{#1}}
\makeatother

\usepackage{natbib}
\usepackage{graphicx}
\usepackage{ifthen}
% Code blocks
\usepackage[outputdir=latex-build]{minted}

\usepackage[table]{xcolor}

% TikZ for Graphs
\usepackage{tikz}
\usepackage{tikz-3dplot}
\usetikzlibrary{positioning, quotes}

% Defining shortcut macros
\newcommand\Z{\mathbb{Z}}
\newcommand\R{{\mathbb{R}}}
\newcommand\Q{{\mathbb{Q}}}
\newcommand\C{{\mathbb{C}}}
\newcommand\N{{\mathbb{N}}}
\newcommand\F{{\mathcal{F}}}
\newcommand\TR{{\operatorname{tr}}}
% Replace \P make paragraph symbol with \P makes \mathcal{P} for power sets
\renewcommand\P{{\mathcal{P}}}
\newcommand{\then}{\Rightarrow}

\DeclareMathOperator{\im}{Im}
\DeclareMathOperator{\arccot}{arccot}

\declaretheoremstyle[
	spaceabove={1em plus 0.1em minus 0.2em},
	notefont=\bfseries,
	notebraces={}{},
	headformat=\ifthenelse{\equal{\NOTE}{}}{\NAME{} \NUMBER}{\!\!\NOTE}
]
{thm}

\declaretheorem{theorem}
\declaretheorem[name=Farka's Lemma, numbered=unless unique]{farkas}
\declaretheorem[style=remark]{example}

\declaretheoremstyle[
	spaceabove={1em plus 0.1em minus 0.2em},
	notefont=\bfseries,
	notebraces={}{},
	headformat=\ifthenelse{\equal{\NOTE}{}}{\NAME{} \NUMBER}{\!\!\NOTE}
]
{problem}

\declaretheorem{problem}[style=problem]

\declaretheoremstyle[
	headfont=\itshape,
	bodyfont={
			% spacing inbetween paragraphs (with rubber padding)
			\parskip 0.3em plus 0.1em minus 0.2em
			% disables paragraph indenting
			\parindent 0pt
			% no obnoxious space before lists
			\setlist{nosep}
			% indent description lists
			\setlist[description]{labelindent=1.4em,labelwidth=!,format=\normalfont\itshape}
		}
]
{solution}
\declaretheorem{solution}[numbered=no, style=solution]

\newcommand{\solpart}[1]{\par\textbf{#1}  }

\title{Linear Programming}
\author{George Ekman}

\begin{document}
\maketitle

\section{The Dual of a Linear Program}

When defining a linear program in terms of $A, b, c$, we actually have two nicely related linear
programs, the primal problem and the dual problem. In standard form,
\begin{align*}
	\textbf{Primal: } & \min c^t x \quad \text{subject to } Ax = b,\ x \ge 0 \\
	\textbf{Dual: } & \max b^t y \quad \text{subject to } A^t y \le c
\end{align*}
with $A$ an $m \times n$ matrix, $b, y \in \R^m$, and $c, x \in \R^n$. In particular, the
optimal values of these two problems are equal, a property called Strong Duality.

Standard form has the primal constraints written as equalities, but a linear program with linear
inequalities in either direction, or even no sign restrictions on $x$, can always be rewritten in
standard form, and solutions and properties will be transferable between the forms. The dual can
also be written as an equality, but it plays better in the coming proofs if it is an inequality.

A linear program is \emph{feasible} if its region of feasibility is nonempty, so there exist
vectors satisfying the constraints, and is \emph{unbounded} if its optimal value does not exist on
the region of feasibility. A linear program with a solution is both feasible and bounded.

\section{Examples}

\begin{example}
	Consider the problem:
	\begin{align*}
		\max x_1 + x_2 + x_3 \quad \text{subject to} \\
		x_1 + x_2 + x_3 = 6 \\
		2x_1 + 3x_2 + x_3 = 8 \\
		2x_1 + x_2 + 3x_3 = 0 \\
		x \ge 0
	\end{align*}

	Written in matrix form:
	\begin{align*}
		\max \begin{bmatrix} 1 \\ 1 \\ 1 \end{bmatrix}^t x \quad \text{subject to} \quad
		\begin{bmatrix}
			1 & 1 & 1 \\
			2 & 3 & 1 \\
			2 & 1 & 3
		\end{bmatrix} x = \begin{bmatrix} 6 \\ 8 \\ 0 \end{bmatrix}, \quad x \ge 0
	\end{align*}

	Written in standard form:
	\begin{align*}
		\min \begin{bmatrix} -1 \\ -1 \\ -1 \end{bmatrix}^t x \quad \text{subject to} \quad
		\begin{bmatrix}
			1 & 1 & 1 \\
			2 & 3 & 1 \\
			2 & 1 & 3
		\end{bmatrix} x = \begin{bmatrix} 6 \\ 8 \\ 0 \end{bmatrix}, \quad x \ge 0
	\end{align*}

	Looking at the first and last constraints in the matrix, along with $x \ge 0$, shows the primal is
	infeasible: $2x_1 + x_2 + 3x_3 = 0$ and $x \ge 0$ means $x = 0$, which does not satisfy $x_1 + x_2
		+ x_3 = 6$.

	And its dual:
	\begin{align*}
		\max \begin{bmatrix} 6 \\ 8 \\ 0 \end{bmatrix}^t y \quad \text{subject to} \quad
		\begin{bmatrix}
			1 & 2 & 2 \\
			1 & 3 & 1 \\
			1 & 1 & 3
		\end{bmatrix} y \le \begin{bmatrix} -1 \\ -1 \\ -1 \end{bmatrix}
	\end{align*}

	Taking $y = -e_1$, we get
	\begin{align*}
		\begin{bmatrix}
			1 & 2 & 2 \\
			1 & 3 & 1 \\
			1 & 1 & 3
		\end{bmatrix} \begin{bmatrix} \-1 \\ 0 \\ 0 \end{bmatrix}
		&= \begin{bmatrix} -1 \\ -1 \\ -1 \end{bmatrix} \\
		&\le \begin{bmatrix} -1 \\ -1 \\ -1 \end{bmatrix}
	\end{align*}
	So the dual is feasible. What about boundedness?
\end{example}

\begin{example}
	Consider the problem:
	\begin{align*}
		\min -x_1 + 2x_2 \quad \text{subject to} \\
		x_1 + x_2 \le 10 \\
		x_1 + 2x_2 \le 12 \\
		x \ge 0
	\end{align*}

	Written in matrix form:
	\begin{align*}
		\min \begin{bmatrix} -1 \\ 2 \end{bmatrix}^t x \quad \text{subject to} \quad
		\begin{bmatrix}
			1 & 1 \\
			1 & 2 \\
		\end{bmatrix} x \le \begin{bmatrix} 10 \\ 12 \end{bmatrix}, \quad x \ge 0
	\end{align*}

	Written in standard form:
	\begin{align*}
		\min \begin{bmatrix} -1 \\ 2 \end{bmatrix}^t x \quad \text{subject to} \quad
		\begin{bmatrix}
			1 & 1 & 1 & 0 \\
			1 & 2 & 0 & 1 \\
		\end{bmatrix} \begin{bmatrix} x \\ s \end{bmatrix} = \begin{bmatrix} 10 \\ 12 \end{bmatrix},
		\quad x, s \ge 0
	\end{align*}

	Taking $x = 0$ ($s = \begin{bmatrix} 10 & 12 \end{bmatrix}^t$) shows this problem is feasible.

	The dual:
	\begin{align*}
		\max \begin{bmatrix} 10 \\ 12 \end{bmatrix}^t y \quad \text{subject to} \quad
		\begin{bmatrix}
			1 & 1 \\
			1 & 2 \\
			1 & 0 \\
			0 & 1 \\
		\end{bmatrix} y \le \begin{bmatrix} -1 \\ 2 \\ 0 \\ 0 \end{bmatrix}
	\end{align*}

	Again, taking $y = -e_1$ shows this problem is feasible. What about boundedness?
\end{example}

From the examples, it is not to hard to determine if a problem is feasible, but it is difficult to
determine if it is bounded, and this is important because a problem must be both feasible and
bounded for a solution to exist. This is where the connection between a primal problem and its dual
come in: a primal is bounded if its dual is feasible (Weak Duality), and if both are feasible, or
one has a solution, they both have the same optimal value (Strong Duality).

To prove this, we'll first need to lay some groundwork connecting the primal and dual problems
before connecting their feasibility and boundedness.

\section{Farka's Lemma}

\begin{theorem}
	\label{thm:funlinalg}%
	Either there exists $x$ such that $Ax = b$, or there exists $y$ such that $y^t A = 0$ and
	$y^t b \neq 0$.
\end{theorem}

Farka's Lemma is a version of this theorem, but tailored for linear programming, providing a
connection between the primal and dual problems, since $A^t y \ge 0 \iff y^t A \ge 0$.

\begin{farkas}
	Exactly one is true:
	\begin{enumerate}
		\item There exists $x \ge 0$ such that $Ax = b$
		\item There exists $y$ such that $y^t A \ge 0$ and $y^t b = -1$.
	\end{enumerate}
\end{farkas}

The first case means the program is feasible, with $x$ a feasible solution, and the second case
means it is infeasible, with $y$ being the \emph{certificate of infeasibility}. Geometrically, $Ax$
with $x \ge 0$ creates a positive cone $C$, which either contains $b$ (the first case), or we can
find a hyperplane through the origin defined by $y$, where the cone is above the hyperplane but $b$
is below it.

Apart from the geometric meaning talking about feasibility of the primal problem, Farka's Lemma
provides a way to convert statements about the primal into statements about the dual, which we will
use in the proofs about duality.

\begin{proof}
	Suppose the first case holds, there exists an $x \ge 0$ such that $Ax = b$. For any $y$ such that
	$y^t A \ge 0$, $y^t A x \ge 0$, which means $y^t b \neq -1$, and so the second case fails.

	Suppose instead that the first case fails. If we define $C = \left\{ Ax : x \ge 0 \right\}$, this
	means $b \not\in C$.

	Now, we need to construct a $y$ that satisfies the second case. We will do this by find the closest
	vector $w$ to $b$, considering the vector $z$ from $b$ to $w$ so that $y^t A \ge 0$ and $y^t b <
		0$, and then rescaling it so $y^t b = -1$.

	This set $C$ is closed and convex, so if we intersect it with $B = \{ x : \Vert b - x \Vert \le
		\Vert b \Vert \}$, the closed ball of radius $\Vert b \Vert$ around $b$, we have a compact and
	convex set $C \cap B$, nonempty since it contains at least $0$. Therefore, the strictly convex
	$\Vert b - x \Vert$ has a unique minimizer $w \in C \cap B$, the closest vector in $C$ to $b$.

	For any $v \in C$, $\lambda \in [0, 1]$, since $C$ is convex and $w$ is closest to $b$,
	\begin{align*}
		\Vert w - b \Vert^2 &\le \Vert (1 - \lambda)w + \lambda v - b \Vert^2 \\
		&\le \Vert \lambda(v - w) + (w - b) \Vert^2 \\
		&\le \lambda^2 \Vert v - w \Vert^2 + 2 \lambda \langle v - w, w - b \rangle + \Vert w - b \Vert^2 \\
		0 &\le \lambda^2 \Vert v - w \Vert^2 + 2 \lambda \langle v - w, w - b \rangle
	\end{align*}

	Post image And so for small $\lambda$, $\langle v - w, w - b \rangle \ge 0$.

	Define $z = w - b$, the vector from $b$ to $w$. This inequality now says $\langle v - w, z \rangle
		\ge 0$, and when $v = 0$, it gives $\langle w, z \rangle \le 0$. Then, looking at $\langle b, z
		\rangle$,
	\begin{align*}
		\langle b, z \rangle &= \langle b - w, z \rangle + \langle w, z \rangle \\
		&= - \Vert z \Vert^2 + \langle w, z \rangle \\
		&< 0
	\end{align*}

	For all $v \in C$, $\langle v - w, z \rangle \ge 0$ also means that $\langle v, z \rangle \ge
		\langle w, z \rangle$, and with some $\gamma < 0$, we can combine our inequalities in a chain:
	\[ \langle v, z \rangle \ge \langle w, z \rangle > \gamma > \langle b, z \rangle \text{.} \]

	If we fix $v \in C$, then for all $\lambda > 0$, the chain gives $\langle \lambda v, z \rangle >
		\gamma$, and so $\langle v, z \rangle > \gamma / \lambda$, and as $\lambda \to \infty$, $\gamma /
		\lambda \to 0$, which means we can rewrite our chain with $\gamma = 0$,
	\[ \langle v, z \rangle \ge 0 > \langle b, z \rangle \text{,} \]
	Leaving us only to rescale by defining $y = \frac{z}{|\langle z, b \rangle|}$ so that
	\[ \langle v, y \rangle \ge 0 > \langle b, y \rangle = -1 \text{.} \]
	Which means $y^t A \ge 0$ and $y^t b = -1$.
\end{proof}

\section{Weak Duality}

\begin{theorem}[Weak Duality]
	If $x$ is feasible for the primal problem, and $y$ is feasible for the dual problem, then
	$c^t x \ge y^t b$.

	In particular, this means that if one is feasible, the other is bounded.
\end{theorem}
\begin{proof}
	$c^t \ge y^t A$, so $c^t x \ge y^t A x \ge y^t b$.
\end{proof}

Modifying the constraints or changing between maximizing and minimizing will change the exact
statement of Weak Duality (flipping inequalities), so it is useful to think about Weak Duality in
the second way.

For example, if the primal problem is feasible with $x$, the dual is bounded, since for any
feasible $y$, $y^t b \le c^t x$. However, in this case, it could be that the dual is infeasible,
and there exists no such $y$.

\section{Strong Duality}

\begin{theorem}[Strong Duality]
	If both problems are feasible, then they have the same objective value: $c^tx_* = b^ty_*$, where
	$x_*$ and $y_*$ are respectively the optimal primal and dual solutions.

	\noindent
	In addition, this means the primal is feasible and bounded if and only if the dual is feasible and
	bounded.
\end{theorem}
\begin{proof}
	Suppose both the primal and dual problems are feasible. Then, by Weak Duality, both problems are
	bounded.

	Since the primal problem is feasible and bounded, there exists an optimal solution $x_*$. Since
	$x_*$ is optimal, this means there does not exist an $x \ge 0$ such that it satisfies the
	constraints, $Ax = b$, and has a lesser optimal value, $c^tx < c^tx_*$. Using $\epsilon > 0$, this
	inequality can be written as the equality $c^tx = c^tx_* - \epsilon = \gamma$. Then, writing this
	as a linear program,
	\[
		\nexists x \ge 0 \in \R^n \text{ such that }
		\begin{bmatrix} A \\ c^t \end{bmatrix} x = \begin{bmatrix} b \\ \gamma \end{bmatrix}
		\text{.}
	\]
	Applying Farka's Lemma gives us that since the first case fails, the second must hold:
	\[
		\exists \begin{bmatrix} -y \\ \alpha \end{bmatrix} \in \R^{m + 1} \text{ such that }
		\begin{bmatrix} A^t & c \end{bmatrix} \begin{bmatrix} -y \\ \alpha \end{bmatrix} \ge 0
		\text{ and }
		\begin{bmatrix} b^t & \gamma \end{bmatrix}\begin{bmatrix} -y \\ \alpha \end{bmatrix} = -1
		\text{.}
	\]
	The sign in $-y$ makes the end of the proof cleaner. To write this back out as constraints, we have
	\begin{align*}
		-A^t y + \alpha c &\ge 0 \\
		-b^t y + \alpha \gamma &= - 1
	\end{align*}
	We want to play with these constraints to show first that the dual is feasible, and then that its
	optimal value must be equal to the primal's. To do this, we will need to remove the $\alpha$, and
	we can do this by considering two cases, $\alpha = 0$ and $\alpha \neq 0$.

	Suppose $\alpha = 0$. Then,
	\begin{align*}
		A^t (-y) &\ge 0 \\
		b^t (-y) &= - 1
	\end{align*}
	But this shows there exists $-y$ satisfying the second case of Farka's Lemma, which would mean the
	primal is infeasible. However, we know the primal is feasible, so we know $\alpha \neq 0$.

	Now that $\alpha \neq 0$, we can scale our constraints by $\alpha^{-1}$ so that the $\alpha$ drop
	out. To keep the notation clean, this means we can just assume $\alpha = 1$, and keep using $y$:
	\begin{align*}
		-A^t y + c &\ge 0 \\
		-b^t y + \gamma &\le 0
	\end{align*}
	That last equality is written as an inequality to remove the $\alpha^{-1}$ from the right. These
	constraints can be rewritten, and here is why I used $-y$,
	\begin{align*}
		A^t y &\le c \\
		b^t y &\ge \gamma
	\end{align*}
	This means $y$ is feasible for the dual problem, so the optimal value of the dual is also bounded
	below by $\gamma$ since we are maximizing: $b^ty_* \ge b^t y \ge \gamma$. Then, combining that last
	inequality with the bound from Weak Duality,
	\[ c^tx_* \ge b^t y_* \ge \gamma = c^tx_* - \epsilon \]
	And so finally, as $\epsilon \to 0$,
	\[ c^tx_* = b^t y_* \text{.}\]
\end{proof}

At the beginning of the proof we assumed that both the primal and dual were feasible, and then used
Weak Duality to show that the dual being feasible meant the primal was bounded, and then used the
primal problem being feasible and bounded to get the optimal $x_*$. We could instead have started
with the primal being both feasible and bounded, and not assumed anything about the dual, to get
$x_*$. Then we could have used feasibility of the primal to get boundedness of the dual, and we
construct a feasible $y$ for the dual in the proof so we know the dual is bounded.

This argument works both ways, so the primal is feasible and bounded if and only if the dual is
bounded.

\end{document}
