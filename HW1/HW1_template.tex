% This is a template for doing homework assignments in LaTeX, cribbed from M. Frenkel (NYU) and A. Hanhart (UW-Madison)

\documentclass{article} % This command is used to set the type of document you are working on such as an article, book, or presenation

\usepackage[margin=1in]{geometry} % This package allows the editing of the page layout. I've set the margins to be 1inch.

\usepackage{amsmath, amsfonts}  % The first package allows the use of a large range of mathematical formula, commands, and symbols.  The second gives some useful mathematical fonts.

\usepackage{graphicx}  % This package allows the importing of images

%Custom commands
\newcommand{\der}[2]{\frac{\mathrm{d} #1}{\mathrm{d} #2}}
\newcommand{\pder}[2]{\frac{\partial #1}{\partial #2}}

%Custom symbols
\newcommand{\Rb}{\mathbb{R}}
\newcommand{\Nb}{\mathbb{N}}
\newcommand{\Zb}{\mathbb{Z}}
\newcommand{\Qb}{\mathbb{Q}}
\newcommand{\Cb}{\mathbb{C}}

\begin{document}

\begin{center}
	\Large{\textbf{Assignment \#1}

		Summer 2022 REU} \vspace{5pt} % Name of course here

	\normalsize{ George Ekman

		\today}        % Change to due date if preferred
	\vspace{15pt}

\end{center}

\subsection*{Exercise One:}

a) Code:
\begin{verbatim}
If $f(x) = x^n$,  then
$$
f^\prime(x) = n x^{n-1}.
$$
\end{verbatim}

Output: If $f(x) = x^n$, then $$ f^\prime(x) = n x^{n-1}. $$

\noindent b)
\begin{verbatim}
	If $n \neq -1$, then
	\[ \int x^n dx = \frac{1}{n + 1} x^{x + 1} + C \text{.} \]
\end{verbatim}
Output: If $n \neq -1$, then \[ \int x^n dx = \frac{1}{n + 1} x^{x + 1} + C \text{.} \]

\noindent c)
\begin{verbatim}
	The derivative of a function $f$ at $x = a$ is
	\[ f'(a) = \lim_{h \to 0} \frac{f(a + h) -f(a)}{h} \text{.} \]
\end{verbatim}
Output: The derivative of a function $f$ at $x = a$ is
\[ f'(a) = \lim_{h \to 0} \frac{f(a + h) -f(a)}{h} \text{.} \]

\subsection*{Exercises Two:}
\noindent a)
\begin{verbatim}
The number $e$ is defined by
\[ e = \lim_{n \to \infty} \left(1 + \frac{1}{n}\right)^n \text{.} \]
\end{verbatim}
Output: The number $e$ is defined by
\[ e = \lim_{n \to \infty} \left(1 + \frac{1}{n}\right)^n \text{.} \]

\noindent b)
\begin{verbatim}
If $f$ is a continuous function, then
\[ \frac{d}{dx} \left[ \int_a^x f(t)dt \right] = f(x) \text{.} \]
\end{verbatim}
Output: If $f$ is a continuous function, then
\[ \frac{d}{dx} \left[ \int_a^x f(t)dt \right] = f(x) \text{.} \]

\subsection*{Exercise Three:}
Code:
\begin{verbatim}
\begin{center}
	\begin{tabular}{|rl|cc|}
		\hline
		First Name & Last Name & Ice Cream Flavor & Number of Scoops \\
		\hline
		George     & Ekman     & Butter Pecan     & $\infty$         \\
		Alexa      & Leal      & Vanilla          & $4$              \\
		Julia      & Maschi    & Chocolate        & $2$              \\
		Johnny     & Tran      & Strawberry       & $18$             \\
		\hline
	\end{tabular}
\end{center}
\end{verbatim}
Output:
\begin{center}
	\begin{tabular}{|rl|cc|}
		\hline
		First Name & Last Name & Ice Cream Flavor & Number of Scoops \\
		\hline
		George     & Ekman     & Butter Pecan     & $\infty$         \\
		Alexa      & Leal      & Vanilla          & $4$              \\
		Julia      & Maschi    & Chocolate        & $2$              \\
		Johnny     & Tran      & Strawberry       & $18$             \\
		\hline
	\end{tabular}
\end{center}

\subsection*{Exercise Four:}
Code:
\begin{verbatim}
\[ \det \begin{pmatrix} a & b \\ c & d \end{pmatrix} = ad - bc \text{.} \]
\end{verbatim}
Output:
\[ \det \begin{pmatrix} a & b \\ c & d \end{pmatrix} = ad - bc \text{.} \]

\subsection*{Exercise Five:}
Code:
\begin{verbatim}
% In preamble:
\newcommand{\pder}[2]{\frac{\partial #1}{\partial #2}}

\newcommand{\Nb}{\mathbb{N}}
\newcommand{\Zb}{\mathbb{Z}}
\newcommand{\Qb}{\mathbb{Q}}
\newcommand{\Cb}{\mathbb{C}}

% Here:
\[ \Nb \subset \Zb \subset \Qb \subset \Rb \subset \Cb \text{.} \]

If $z = x^2 + xy + y^2$, then
\[ \pder{z}{x} = 2x + y \text{.} \]
\end{verbatim}

Output:
\[ \Nb \subset \Zb \subset \Qb \subset \Rb \subset \Cb \text{.} \]

If $z = x^2 + xy + y^2$, then
\[ \pder{z}{x} = 2x + y \text{.} \]

\subsection*{Exercise Six:}
Code:
\begin{verbatim}
If $f(x) = x^2$, then
\begin{align*}
	f'(a) &= \lim_{h \to 0} \frac{f(a + h) - f(a)}{h} \\
		  &= \lim_{h \to 0} \frac{(a + h)^2 - a^2}{h} \\
		  &= \lim_{h \to 0} \frac{a^2 + 2ah + h^2 - a^2}{h} \\
		  &= \lim_{h \to 0} 2a + h = 2a \text{.}
\end{align*}
\end{verbatim}
Output: If $f(x) = x^2$, then
\begin{align*}
	f'(a) &= \lim_{h \to 0} \frac{f(a + h) - f(a)}{h} \\
	&= \lim_{h \to 0} \frac{(a + h)^2 - a^2}{h} \\
	&= \lim_{h \to 0} \frac{a^2 + 2ah + h^2 - a^2}{h} \\
	&= \lim_{h \to 0} 2a + h = 2a \text{.}
\end{align*}

\end{document}
